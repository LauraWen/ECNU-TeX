% !TeX document-id = {32a890c7-0db2-461b-b834-605db552d4de}
% !TEX root
% !TEX program = xelatex
% !BIB program = biber

\documentclass[a4paper,zihao=-4,UTF8,punct,linespread=1.56]{ctexart}
\pagestyle{empty} % 第二页以后页码空白
\usepackage[a4paper,left = 3.2cm, right = 3.2cm, top = 2.54cm, bottom = 2.54cm]{geometry}
\usepackage{xcolor}
\usepackage{setspace}
%\usepackage{xltxtra,mflogo,texnames}
\usepackage[colorlinks]{hyperref}  


%\usepackage[backend=biber,style=gb7714-2015,gbnamefmt=givenahead,
%gbpub=false,gbbiblabel=dot,gbtitlelink=true]{biblatex}%sorting=nyt

\usepackage[ 
	backend		=	biber,
	style		=	gb7714-2015mx, 
	% Chinese support bibliography style gb7714-2015/gb7714-2015ay/gb7714-2015mx
	gbnamefmt	=	givenahead,
	gbnamefmt 	= 	lowercase, 
	giveninits 	= 	true,
	maxbibnames	=	99
]{biblatex}
%使用biblatex管理文献,输出格式使用gb7714-2015标准,后端为biber
\DeclareDelimFormat{finalnamedelim}{,}

\addbibresource{./reference/refs.bib}
\addbibresource{./reference/thesis-ref.bib}

\renewcommand*{\bibfont}{\small}
\setlength{\bibitemsep}{0ex}
\setaystylesection{2}

\begin{document}
	
\title{GB7714-2015文献测试}	
\author{汤银才}

\maketitle

\section{说明}

本文测试

\begin{enumerate}
	\item biblatex文献管理. 文献生成通过biber 代替原有的bibtex生成参考文献;
	\item 测试由hushidong 制作的符合 国标GB/T7714-2015 标准的 biblatex 样式
\end{enumerate}

参考文献的生成
\begin{itemize}
	\item 需要保证bib文献库格式的规范正确, 可用文本或Jabref软件生成;示例中给出了常用的文献式样. 	
	\item 编译过程
	\begin{verbatim}
	xelatex 文件名
	biber 文件名
	xelatex 文件名	
	xelatex 文件名
	\end{verbatim}
\end{itemize}


\section{参考文献基本流程}
\begin{enumerate}
\item 准备: 导言处通过命令 addbibresource 添加后缀为bib的文献库,一次添加一个,可添加多个. 

\begin{verbatim}
\addbibresource{./reference/refs.bib}
\end{verbatim}

\item 正文中通过命令 printbibliography 生成.	

\begin{verbatim}
\printbibliography[
    title={\centerline{\bfseries\sffamily \zihao {-3}参考文献}}]
\end{verbatim}

\end{enumerate}

\section{参考文献生成1}
\begin{refsection}

\subsection{引用}

\begin{enumerate}
	\item  上标引用
	
	
	胡伟\cite{Wuwei:2013}
	
	\authornumcite{Wuwei:2013}
	
	\item 正常引用
	
	胡伟\parencite{Wuwei:2013},
	
	\textcite{Wuwei:2013}
\end{enumerate}



\nocite{*}

\begin{spacing}{1.2} % 行距
\zihao{5} \songti
\printbibliography[heading=subbibliography, 
	  title={\centerline{\bfseries\sffamily \zihao {-3}参考文献}}]
\end{spacing}	
			   
\end{refsection}

\section{参考文献生成2}
\begin{refsection}

\subsection{引用}

	
	贝叶斯\cite{Bayes63:classical}
	
		
	贝叶斯\parencite{Bayes63:classical},
	
	\textcite{Bayes63:classical}

\nocite{BEZOS02,ELIDRISSI94,MELLINGER96,SHELL02,Joa:1999,Yang_Hy200215,Xie_dy1997,Deng_PC2001,Shang_dy2001}

\begin{spacing}{1.2} % 行距
	\zihao{5} \songti
	\printbibliography[heading=subbibliography,
	title={\centerline{\bfseries\sffamily \zihao {-3}参考文献}}]
\end{spacing}				   
\end{refsection}


\end{document}